\documentclass[sigconf, nonacm, review]{acmart}
\renewcommand\footnotetextcopyrightpermission[1]{} 
\settopmatter{printacmref=false}
\pagestyle{plain}
\begin{document}


\title{The title}
\subtitle{Accelerator-Centric Computing Ecosystems -- Programming Assignment Report }

%%
%% The "author" command and its associated commands are used to define
%% the authors and their affiliations.
%% Of note is the shared affiliation of the first two authors, and the
%% "authornote" and "authornotemark" commands
%% used to denote shared contribution to the research.


\author{Author 1}
\email{author1@mail.com}
\affiliation{%
  \institution{Computer Science, VU Amsterdam}
  \country{}
}

\author{Author 2}
\email{author2@mail.com}
\affiliation{%
  \institution{Computer Science, VU Amsterdam}
  \country{}
}
\author{Author 3}
\email{author3@mail.com}
\affiliation{%
  \institution{Computer Science, VU Amsterdam}
  \country{}
}


%%
%% The abstract is a short summary of the work to be presented in the
%% article.
\begin{abstract}
  [Optional]  a high-level description of your solution, analysis overview, and main results
\end{abstract}




%%
%% This command processes the author and affiliation and title
%% information and builds the first part of the formatted document.
\maketitle

\section{Introduction}
A brief overview of the problem (K-means clustering), motivation for FPGA acceleration, and the structure of the remainder of the article. Use one short paragraph for each.\\

\noindent
\textit{Note: you are allowed to deviate from this template as long as the same information is included and it is clear where it is discussed.}


\section{Implementation design}
In this section, you should explain the design of your implementation(s) for Vitis HLS. As the assignment is split into two parts, we expect you to discuss (but not limited to):
\begin{itemize} 
\item For the base implementation: detail the modifications required to port the CPU-based code to Vitis HLS. For instance, discuss any unsupported constructs you encountered (e.g., dynamic memory allocation) and explain how you resolved or replaced them to ensure compatibility with Vitis HLS synthesis.
\item For the optimized implementation (if completed): 
Describe the optimizations you applied to improve performance. Explain the techniques used, how they were implemented, and the reasoning behind each choice. 
\end{itemize}

\section{Implementation details}
If not already included in the previous section.

\section{Evaluation}

The quality of your solution is evaluated by the \textit{latency}, \textit{resource consumption}, and \textit{synthesis frequency} (Fmax) reported after the synthesis step.\\ 

\noindent
We expect you to analyze the report and discuss the metrics above. Specifically, reason about the performance (latency) of the various loops in the top function.  Aspects to discuss include (but are not limited to): \textit{Are the loops pipelined? What is the achieved Initiation Interval (II)? Why is the II not equal to 1, if that's the case?}\\

\noindent
If you also implement an optimized version of the K-Means algorithm, clearly highlight how your optimizations affected performance and resource usage. Describe what changed, why, and how it contributed to improvement (or if it introduced trade-offs).\\

\noindent
Since performance characteristics may vary depending on the input dataset, we encourage you to evaluate and compare results across multiple datasets.


\section{Discussion and Conclusion}

Describe the challenges, limitations, and possible optimizations to your solution. We expect you to provide a summary of your findings and provide a comparison between the FPGA-based and GPU-based implementations of the K-Means algorithm, focusing not only on performance differences but also on development complexity and design constraints.  Finally, (optional but encouraged) you can include lessons learned and a time sheet reporting the time it took to conduct each major part of the assignment.


%% The next two lines define the bibliography style to be used, and
%% the bibliography file.
\bibliographystyle{ACM-Reference-Format}
\bibliography{sample-base}
\end{document}